%!TEX root = vaisagh_thesis.tex

\begin{abstract}

Crowd simulation is gaining an increasing amount of importance in recent years for a variety of purposes ranging from movies and games to safe design of buidlings, planning for major events like the Olympics and for preparing for emergencies. Over the last few decades, the crowd simulation models have come a long way from the simpler network based approaches to models using agents based simulation which has a much greater amount of detail. The bottom up approach of agents based simulation allows modelers to consider the complexity of human behavior in much more detail than was traditionally feasible. It is possible to consider thousands of individuals with their individual behavior characteristics occupying and interacting in complex indoor environments like shopping malls.

However, existing computational models fail to take into consideration the large body of work on human behavior during emergencies and make unsubstantiated assumptions like perfect knowledge of the layout, existence of panic and immediate evacuation on hearing a fire alarm. In this thesis, we study the existing work on human behavior and identify perception, event identification, spatial knowledge acquisition and navigation as the four key building blocks of the behavior model to be used in an agent based simulation of emergency egress. Following this, key shortcomings in existing models of each of these are identified and addressed. It is first demonstrated how a simple extension to existing perception systems based on the idea of the human's having a limited information processing capacity can be used to produce more realistic motion from existing motion planning systems. Following this, a novel event identification system for modelling pre-evacuation behavior is introduced and used to demonstrate the importance of considering pre-evacuation behavior when preparing for emergencies. Next, wayfinding behavior in multi floor indoor environments is studied and analysed through a game developed using Minecraft. Finally, a method for quantitatively comparing motion planning systems is proposed and demonstrated by comparing three of the most popular ones.
\end{abstract}
