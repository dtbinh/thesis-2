%!TEX root = thesis.tex

\begin{abstract}
Crowd egress is an area of research that has increasingly become more popular. Research over the past few decades have increased the current understanding of human and crowd behavior by leaps and bounds. During the same period, advances in technology and experience in modeling and simulation have lead to increasingly capable modeling and simulation systems.

This report presents the preliminary work done towards the author's thesis and describes some of the details of the proposed computational model of egress. A study of existing literature on crowd, egress and general human behavior reveals that, while there are a large number of theories of human behavior, there are a few characteristics of human behavior that have become well established. The fact that humans do not behave irrationally during a fire evacuation; rather, they behave rationally within the bounds of the information they have is a rather surprising one. Other salient features of egress behavior include the fact that there is a significant pre-evacuation period where the evacuee completes the task he/~she is doing, searches for more information to clarify the danger of the situation and finds his/~her primary group before they actually start the process of moving towards their preferred exit. The proposed model proposes to model these \emph{salient features} of human egress behavior.

The purpose of this thesis, is to make use of most current knowledge of human behavior in creating a detailed Agent Based Model and Simulation of Crowd Egress from a building on fire. The model is based on the idea that humans are serial information processors who use the information about the world around them as much as possible to evacuate. This information can be in the form of obstacles that he/~she perceives; it can be events that he observes; it can refer to messages that are communicated between agents and it also includes information stored in the person's cognitive map to plan a route for escape. However, the limitations in information processing capability of the human brain and the experience and background of the evacuee makes it difficult for him/~her to know the best route, remember the nearest exit and perceive every single clue that might help him/~her. These are complications that are seldom considered in existing computational models of egress.


\end{abstract}
