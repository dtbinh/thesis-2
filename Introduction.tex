%!TEX root = vaisagh_thesis.tex

\chapter{Introduction}
\label{chapter:Introduction}

% What is the problem? The thesis paper topic;
% The reasons which pushed a student to write his or her thesis paper exactly on this topic; Why is it interesting and important?
% The thesis topic preface, or the background information on the thesis paper topic; Why is it hard? (E.g., why do naive approaches fail?)
% Why hasn't it been solved before? (Or, what's wrong with previous proposed solutions? How does mine differ?)




At about 3.15 pm on December 30, 1903, the full capacity crowd at the Iroquois theater in Chicago were just getting prepared for the second act of \emph{Mr. Bluebeard}, when tragedy struck. An arc light shorted out and sparks ignited the curtain and started a fire that quickly engulfed the building. 605 deaths were reported in what is still the deadliest single building fire in United States history. Subsequent analysis revealed that the death toll would have been much less had the building design been less confusing. One of the most analyzed among recent such tragedies is the Love Parade disaster of 2010 which occurred in Duisburg, Germany. A crowd rush in the festival area lead to 21 deaths and injuries to more than 500 people. Helbing and Mukerji~\cite{Helbing:2012} concluded that the major reason for the death and destruction was the inadequate capacity of the holding area and the fact that the parade area could only be entered and exited through one tunnel. More than a century has passed since the Iroquois Theater Fire, but stampedes and deaths from people trying to escape from enclosed spaces are still very common and still seem to be caused by inadequacies in planning for emergencies.

Several of these deaths during egress are preventable with proper planning and careful construction of buildings. Modelling and simulation is one of the primary tools through which planning is done these days. For example, the Sydney Olympics made use of the crowd simulation software, LEGION~\cite{Still:2000tp} to test the facilities for their ability to accommodate crowds and handle emergency evacuations. In fact, crowd simulation is an area that has been the subject of a significant amount of multidisciplinary work over the last few decades~\cite{Still:2000tp,Zhou:2010:CMS:1842722.1842725,Gwynne:1999vi}. Its applications range from simulating crowds for movies~\cite{Regelous:2011vt,Reynolds:1987vm} and games~\cite{Snape:2012,ageOfEmpires:2013} to analyzing pedestrian behavior~\cite{Helbing:1995ie,Viswanathan:ut,Guy:2010uv} and preparing for fire evacuations and similar emergencies~\cite{Klupfel:2005to,PEDFull:2011,Mordvintsev:2012}. Animation of large crowds in the award winning movie, The Lord of the Rings was done using a commercial crowd simulation software called MASSIVE~\cite{Regelous:2011vt}. Despite the range of uses, some of the most frequent users of crowd simulation systems are civil defense authorities, who make use of these simulations to study, evaluate and formulate strategies for controlling crowds and for tackling emergencies that can emerge like in the case of the Sydney Olympics. While modelling crowds, or more specifically, crowds engaging in egress, the structure of the building, the spread of fire or smoke and their effect and, most importantly, the behavior and reactions of the evacuees need to be considered. This makes it an inherently difficult system to simulate.

% LEGION is neither the latest nor the most detailed computational model of egress. Computational models of egress have come a long way since the Henderson's Flow model~\cite{HendersonFlow} which, like several models that followed, only simulated the macro dynamics of motion. More recent models attempt at simulating behavior and emotions as well. Pelechano's MACES model integrated an existing popular psychological model of human beings. However, despite the complexity of the behavior model, effects of human cognitive limits  which result in partial spatial knowledge and pre-evacuation delay are not considered in this model. In the rare cases like~\cite{escapes, pelechano} where this is considered, the models are simplified to such an extent that the effect of these cognitive limits on evacuation cannot be analyzed with these models. This thesis attempts to advance the state of the art in the computational modelling and simulation of egress through more realistic behavior models that take into consideration the cognitive limits of human beings and the primary role played by information processing in this.




Recently, in various fields, there has been an increasing emphasis on the importance of recognizing the \emph{complexity} of real world systems~\cite{Arthur:2010uy}. Effort is being put into modelling these complexities to better understand and predict what happens in the real world. In recent times, one of the most popular methodologies for tackling this complexity has been through the use of Agent Based Modelling (ABM). An Agent Based Model is made up of multiple heterogeneous intelligent entities known as agents interacting in an environment. The bottom up approach of modelling each agent's behavior often allows the modeler to implement existing theories of individual behavior directly without having to abstract away too many details. This approach is useful because it is easier to study and formulate theories at an individual level; apply this theory to model a real world system populated by many such individuals; and finally simulate it. Higher level patterns that emerge from this microlevel modelling can be analyzed and studied to learn more about the system and make predictions. Thus agent based models are ideal for models where there is enough information about the behavior of individual interacting entities.


A crowd engaging in egress, with its complex system of several thinking, feeling people, a building structure that shapes their movements and other entities like fires, smoke and fire alarms,  is one such real world system where the complexity of the situation needs to be considered but is often ignored for simplicity.  Managing egress is one of the key challenges faced during all kinds of disasters ranging from fires to terrorist attacks to natural disasters like hurricanes and tsunamis that seem to be becoming increasingly common today. The first step towards managing such situations is understanding how people behave when confronted with these situations. Several studies in psychology and sociology over the past decades have helped us understand a lot more about human behavior during egress than we did even a few years ago. Perhaps, the most interesting (and surprising) among these, is the fact that irrational \emph{mass panic}, that is found in news reports of incidents, does not seem to occur and definitely does not spread through a crowd~\cite{Kobes:2009jx,Schadschneider:2008cz,Reicher:2008ep,Torres:2010tj,Paulsen:1984ti,Sime:1983uy}. The general consensus in the scientific community today is that crowds do not start behaving irrationally when in such situations and generally try their best to help others. Most, if not all, people behave rationally within the bounds of what knowledge they have of the situation in a phenomenon referred to as \emph{bounded rationality}. The assumption of irrational panic is just one among several reasons for many psychologists and sociologists remaining unconvinced about the efficacy and accuracy of the results produced by today's simulations~\cite{Aguirre:2004tn,Torres:2010tj,Sime:1995uu}. Several details of human behavior are  abstracted away for simplicity, when in fact they play a key role in determining the egress dynamics.

Studies emphasizing the importance of bounded rationality and the post-hoc analysis of disasters like those in Iroquois theater and the Love Parade seem to emphasize the important role played by information and the human brain's ability to process information. The limited capacity of the human brain to process information~\cite{Miller:1956tr} at any given time can have a significant effect on the way humans perceive the environment and form their knowledge of the current situation. In fact, studies~\cite{Ozel:2001tn,Davis01122009} have shown that this capacity is reduced further by circumstances like stress, time constraints or old age. This can explain the apparent irrational behavior that is observed in high stress situations like emergency egress. The primary cause of deaths in most crowd disasters, as will be discussed in the subsequent chapters, is the lack of \emph{information}, either about the current situation or the best routes for escape or some other knowledge that might be crucial for safe egress.

Models need to consider all aspects and all phases of egress, from the initial recognition of an emergency all the way to exiting to safety. The research in this thesis takes a holistic view of the entire process and identifies the deficiencies in current modelling and understanding. For example, occupants of a building rarely start evacuating at the sound of the fire alarm. The first reaction of most people is either to ignore the alarm, or at best, try to finish what they are currently doing so that they can investigate and ascertain the need to evacuate. During this investigation period, occupants try to gather more information either by exploring or communicating with other occupants to learn more about the situation. It is only once this investigation confirms the need to evacuate, that actual evacuation starts. There have been incidents like the Boland Hall Fire~\cite{Berry:2000us} where this pre-evacuation delay has caused the deaths of occupants. However, except in rare cases like the ESCAPES model~\cite{Tsai:2011tz}, pre-evacuation behavior is never modeled. Even in these cases, the model is too simple to analyse the effect of pre-evacuation behavior in much detail.


Once evacuations starts, most models assume that people take some form of the shortest path to get to the exit. However, it may be that not all occupants have complete knowledge of the building and so may not know the best route for safe escape. Limitations of experience and human cognitive capacity mean that, unless special effort is taken, most people will end up taking inefficient routes. In rare cases that partial spatial knowledge is considered~\cite{Pelechano:2006ba}, people are assumed to have eidetic memory, never forgetting a room that is perceived once. It is also common to model agents that do a breadth first search in case there is no knowledge, despite there being a lack of evidence indicating that this is how humans explore unknown indoor environments.


Finally, once a path is chosen, the computational model has to simulate how the evacuee takes the path towards that exit. There are several models of \emph{motion planning} that take very different approaches to simulating this evacuee movement. This, however, makes it very difficult to choose the right model since there does not exist a systematic method for comparing existing ones. In fact, traditionally, motion planning models are only validated against empirical observations of pedestrian movement. Moreover, existing models generally consider motion planning as simply the process of effectively avoiding collisions with other agents that are sensed within a circular or, at best, elliptical sensor range of the agent. However, the limited information processing capacity and the cognitive limits of a human being can have an effect on how collisions are avoided and how people move. Modelling this can help produce more realistic movement behavior from existing models of navigation.

In this thesis, we aim to demonstrate the importance and the usefulness of taking into consideration the inefficiencies in a human's ability to process information in simulations of crowds and, in the process, advance the state of the art in the computational modelling of egress.

% Managing crowd egress from buildings is one of the key challenges faced during all kinds of disasters ranging from fires to terrorist attacks to natural disasters like hurricanes and tsunamis that are becoming increasingly common today. Egress modelling and simulation provides an inexpensive, effective and efficient way to analyze and identify the issues that can affect efficient egress in such situations. Several models of crowds and egress have been developed over the years. However, there are still several areas of crowd behavior that have either not been considered in computational models or haven't been analyzed in sufficient enough detail to inspire confidence in their ability to forecast human behavior during an evacuation. This thesis attempts to advance the state of the art in the agent based simulation of egress by identifying shortcomings in existing approaches and proposing solutions to these.


% Even with all the complexity, detail and meticulousness of these models, many psychologists and sociologists are unconvinced about the efficacy and accuracy of the results produced by these simulations~\cite{Aguirre:2004tn,Torres:2010tj,Sime:1995uu}. This is because even the most popular of these models make certain assumptions about human behavior that stand against evidence obtained over the past few decades through extensive studies in the social sciences and humanities~\cite{Torres:2010tj,Sime:1995uu}. Several details of human behavior are wrongly abstracted away, when in fact they play a key role in determining the egress dynamics.

% One such example of an abstraction often applied to models of crowds is the perception system used by the modeled humans. More often than not, only a simple visual perception system that perceives all other humans and objects within a certain distance to the agent is used. There are two problems with this approach. The first one is obvious; humans have other methods of observing the environment including aural and olfactory perception. The second problem arises because of a limitation of the human brain. It can only process a limited amount of information at any given point of time. This limitation is something that we come across constantly in our everyday life but which we often fail to notice. For example, while sitting engrossed in reading a book it might take a while before we notice someone calling us. It is also because of this same reason, that we are able to listen and understand someone better if we close our eyes and listen. It is also using this same principle that magicians perform their magic tricks without the audience noticing the trick.

% Modelling humans with complete knowledge of the spatial layout of the environment is one such abstraction that is often applied to agent based models of crowds. The limited capacity of the human brain to process information~\cite{Miller:1956tr} at any given time can have a significant effect on the way humans perceive the environment and form their cognitive map and thus their egress route. It can also affect the time taken by a participant to start evacuating because he/she might not know about the fire even if the symptoms are right in front of him/her. This is why it is sometimes mistakenly assumed that people evacuating from a building tend to behave irrationally; while the reality is that they are actually just reacting rationally to the limited information that they have~\cite{Kobes:2009jx,Schadschneider:2008cz,Reicher:2008ep,Torres:2010tj,Paulsen:1984ti,Sime:1983uy}. Ignoring this perceived \emph{irrationality} reduces the reliability of these models.



% Recently, experts in various fields~\cite{Arthur:2010uy} are emphasizing the importance of accepting that most systems in the world are complex systems and the need to model these complexities to have a realistic chance at understanding and predicting what happens in the real world.



% One of the primary reasons for early models like lattice gas models~\cite{Takima:2002wr} and flow models~\cite{Henderson:1974ve} abstracting away details was the  and getting an approximate result in order to get such details as are necessary for that particular application. Advances in hardware have removed many of these constraints. As a result, models have increasingly become more detailed and capable~\cite{Pan:2006vp}.

% There have been tremendous advancements in our knowledge of human behavior over the last few decades. However, the majority of the state of the art computational models make assumptions about human behavior without grounding them in the theories and findings from the social sciences. Being a highly inter-disciplinary field this kind of collaboration is absolutely essential.

\section{Problem Statement}
\label{Intro:ProblemStatement}

Computational models of egress play a key role in furthering our understanding of egress behavior and managing crowds by preventing dangerous situations from emerging. Models and simulations need to be developed that are based on the current understanding of human behavior.

Studies over the past few decades have changed our understanding of human behavior during fire evacuations. However, existing models do not take into consideration the considerable work done in other fields~\cite{Aguirre:2004tn}. Several studies~\cite{Kobes:2009jx,Schadschneider:2008cz,Reicher:2008ep,Torres:2010tj,Paulsen:1984ti,Sime:1983uy} have shown how humans always behave rationally with the limited information that they have and that humans hardly ever panic and behave irrationally.
  However, it is still common to model panic/ irrational behavior in evacuating crowds. Pre-evacuation behavior and the search for information are two general characteristics of the fire evacuation process that are very rarely considered in egress simulation models. The few computational models of egress that do consider pre-evacuation behavior~\cite{Pires:2005gs,Klupfel:2003waa} are rather simple. In other cases, like the models of human motion and exploration used in existing models, there are several different approaches that have been used but a consistent methodology for their evaluation and validation is lacking. Similarly, certain aspects of egress behavior like the effect of partial knowledge have not been studied in much detail. It is necessary to first identify the key components of a behavior model of humans during egress to improve existing models. Following this, the effect of human cognitive limits on these components can be identified and modeled and their effect analyzed.

This thesis aims at advancing the state of the art in computational modelling of egress by developing models and simulations that are based on modern theory and, where possible, experiments that were conducted. This is done by first identifying the key components of a behavior model of egress and subsequently proposing models to identify, simulate and analyze the effect of human cognitive limits on these components. This will help produce more realistic models that help plan for and handle emergency egress better.


% This thesis aims at enhancing the capabilities of existing agent based models of egress by proposing novel methods to evaluate and improve core aspects of existing models, and to model certain important features that haven't been given their due importance in existing models.

% What are the key components of my approach and results? Also include any specific limitations. The goals you are going to achieve;% The tasks to complete in order to attain the goals, or the direction of the thesis research development;

\section{Key Contributions and Scope of This Thesis}
\label{Intro:Contributions}

The key contributions of this thesis are:

\begin{itemize}

	\item \textbf{A multidisciplinary survey and analysis of current literature on fire evacuation and crowd behavior.} This serves two important purposes. Firstly, the multi-disciplinary survey of human behavior research provides a clear idea of our current understanding of human behavior during egress. It also provides a context for the subsequent analysis of existing computational models of egress behavior; this analysis gives a better picture of the shortcomings of existing models in taking into consideration the effects of human cognitive limits on evacuation behavior and motivates the rest of the thesis.

    \item \textbf{A four component breakdown of egress behavior modelling.} The reviewed literature is used as a basis for explaining egress behavior as being a result of four components: Perception, Event Identification, Spatial Knowledge Utilization and Navigation. This breakdown helps identify research problems in each of these components.

	\item \textbf{A novel information based perception system that can model the complexities and limitations of the human perception system.} Existing models tend to model perception as a simple a process of visual sensing where the agents senses other agents in a fixed circular range around it. The proposed model takes into consideration the limitations of our information processing capacity and demonstrates the usefulness of modelling it.

	\item \textbf{A new model for event identification which can be used for simulating pre-evacuation behavior and the process of information seeking.} Studies have repeatedly shown pre-evacuation delay among building occupants who are unconvinced about the need to evacuate. However, this has never been computationally modeled in any detail.  Besides proposing a new model for simulating pre-evacuation information seeking, the importance of modelling pre-evacuation behavior is also demonstrated.

	\item \textbf{A human computation based investigation of the nature of spatially  memory and the role of memory in exploration of multi-storey buildings.} A key issue in modelling the effect of human cognitive limits on spatial knowledge utilization is our limited understanding. Existing methods in experimental psychology are difficult to scale; this makes it difficult to draw general conclusions of human spatial memory. The game based investigation introduced in this thesis provides a scalable alternative to existing methods and reveals interesting insight into the role of memory in indoor wayfinding.

    \item \textbf{A detailed quantitative comparison between popular models for motion planning.} Movement modelling is a key component of any crowd simulation model. However, a primary difficulty in comparing existing models and determining the usefulness of a new method is a lack of a standard method for comparison and analysis. In this thesis, a quantitative comparison is performed between three of today's most popular motion planning models and a key metric for differentiating and analyzing models is identified.


\end{itemize}


\section{Organization of the Report}
\label{Intro:Organisation}

 This thesis is organized as follows. Chapter~\ref{chapter:LiteratureReview} provides a comprehensive review of relevant theories and existing models and also provides a critical analysis of some of the most relevant ones. The following chapter uses the literature reviewed in Chapter~\ref{chapter:LiteratureReview} to identify the key building blocks of human behavior during evacuation. The following chapters identify and solve issues in existing computational models of each of these building blocks.
Chapter~\ref{chapter:IBP} presents an Information processing Based Perception Model and demonstrates and validates its capabilities and working through several experiments. Chapter~\ref{chapter:PreEvacuationBehavior} then presents a new model for event identification which can be used for simulating pre-evacuation behavior and also demonstrates through simulations the importance of modelling pre-evacuation behavior. Chapter~\ref{chapter:SpatialKnowledgeChapter} introduces a novel game-based experimental approach to understanding the impact that memory and non-randomness have in the human exploration process. The analysis from this game is used to study exploration of indoor environments. Finally, Chapter~\ref{chapter:MotionPlannerComparison} addresses one of the key issues in movement modelling in crowds by presenting a detailed methodology for comparison of existing motion planning systems and further providing an analysis of some of the most popular motion planning models used today. Chapter~\ref{chapter:finale} concludes the thesis.

