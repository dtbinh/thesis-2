%!TEX root = thesis.tex

\chapter{Introduction}
\label{chapter:Introduction}





% What is the problem? The thesis paper topic;
% The reasons which pushed a student to write his or her thesis paper exactly on this topic; Why is it interesting and important?
% The thesis topic preface, or the background information on the thesis paper topic; Why is it hard? (E.g., why do naive approaches fail?)
% Why hasn't it been solved before? (Or, what's wrong with previous proposed solutions? How does mine differ?)
Evacuation of crowds from buildings is one of the key challenges faced during all kinds of disasters ranging from fires to terrorist attacks to natural disasters like hurricanes and tsunamis that are becoming increasingly common today. Egress modeling and simulation provides an inexpensive, effective and efficient way to analyze and identify the issues that can affect efficient egress in such situations. Several models of crowds and egress have been developed over the years. However, there are still several areas of crowd behavior that have either not been considered in computational models or haven't been analyzed in sufficient enough detail to inspire confidence in their ability to forecast human behavior during an evacuation. This thesis attempts to advance the capabilities of computational models of egress by introducing and demonstrating some tools that help analyze and compare certain aspects of existing models and also introducing some new models for aspects of egress that are yet to be considered in existing models. 


An evacuating crowd, is a complex system with several interacting elements including people, fire, fire alarms and so forth, each of which can cause different complications in the system.  For example, in many cases, occupants are desensitized from hearing false alarms and often do not start to evacuate until they are completely sure that evacuation is required. At the individual level, the difference caused by this can be as little as a few seconds delay in starting evacuation (or as great as 15 minutes~\cite{FIND_PAPER}). From the perspective of someone looking at the whole fire evacuation system this initial delay in starting evacuation might be negligibly small and he/she might be tempted to ignore this aspect. However, when multiplied over an entire crowd of people that are in a building and taking into consideration the effects that this delay can have on interactions, the effects can be profound. 
Recently, experts in various fields~\cite{Arthur:2010uy} are emphasizing the importance of accepting that most systems in the world are complex systems; emphasizing the need to model these complexities to have a realistic chance at understanding and predicting what happens in the real world. 


In recent times, one of the most popular methodologies for tackling this complexity has been through the use of Agent Based Modeling (ABM). An Agent Based Model is made up of multiple heterogeneous intelligent entities known as agents interacting in an environment. The bottom up approach of modeling each agent's behavior often allows the modeler to implement existing theories of individual behavior directly without having to abstract away too many details into an abstract mathematical formula. This approach is generally very useful because it is easier to study and formulate theories at an individual level; apply this theory to model a real world system populated by many such individuals; and finally simulate its working and correctness. Higher level patterns that emerge from this micro modeling can be analyzed and studied to learn more things about the system and make predictions. Thus agent based models are ideal for models where there is enough information about the behavior of individual interacting entities.

Crowd simulation is just such an area that has been the subject of a significant amount of multidisciplinary work over the last few decades~\cite{Still:2000tp,Zhou:2010:CMS:1842722.1842725,Gwynne1999741}. Its applications range from simulating crowds for movies~\cite{Regelous:2011vt,Reynolds:1987vm} and games~\cite{Snape:2012,ageOfEmpires:2013} to analyzing pedestrian behavior~\cite{PhysRevE.51.4282,Viswanathan:ut,Guy:2010uv} and preparing for fire evacuations and similar emergencies~\cite{Klupfel:2005to,PEDFull:2011,Mordvintsev:2012}. One of the most famous recent applications was in animating large crowds in the award winning movie, Lord of the Rings, which used the commercial software called MASSIVE~\cite{Regelous:2011vt}. Another frequent user of crowd simulation systems is civil defense authorities who make use of these simulations to study, evaluate and formulate strategies for controlling crowds and for tackling emergencies that can emerge. The Sydney Olympics made use of crowd simulation software (LEGION)~\cite{Still:2000tp} to test the facilities for their ability to accommodate crowds and emergency evacuations. Still~\cite{Still:2000tp}, while making the aforementioned LEGION model of crowds, conducted extensive surveys and analyses of videos to accurately model the movement of the crowds. 

Even with all the complexity, detail and meticulousness of these models, many psychologists and sociologists are unconvinced about the efficacy and accuracy of the results produced by these simulations~\cite{Aguirre:2004tn,Torres:2010tj,Sime:1995uu}. This is because even the most popular of these models make certain assumptions about human behavior that stand against evidence obtained over the past few decades through extensive studies in the social sciences and humanities~\cite{Torres:2010tj,Sime:1995uu}. Several details of human behavior are wrongly abstracted away, when in fact they actually play a key role in determining the egress dynamics.

% One such example of an abstraction often applied to models of crowds is the perception system used by the modeled humans. More often than not, only a simple visual perception system that perceives all other humans and objects within a certain distance to the agent is used. There are two problems with this approach. The first one is obvious; humans have other methods of observing the environment including aural and olfactory perception. The second problem arises because of a limitation of the human brain. It can only process a limited amount of information at any given point of time. This limitation is something that we come across constantly in our everyday life but which we often fail to notice. For example, while sitting engrossed in reading a book it might take a while before we notice someone calling us. It is also because of this same reason, that we are able to listen and understand someone better if we close our eyes and listen. It is also using this same principle that magicians perform their magic tricks without the audience noticing the trick.

Modeling humans with complete knowledge of the spatial layout of the environment is one such abstraction that is often applied to agent based models of crowds. The limited capacity of the human brain in processing information at any given time can have a significant effect on the way humans perceive the environment and form their cognitive map and thus their egress route. As explained earlier, it can also affect the time taken by a participant to start evacuating because he/she might not know about the fire even if the symptoms
oms are right in front of him/her. This is why it is sometimes mistakenly assumed that people evacuating from a building tend to behave irrationally; while the reality is that they are actually just reacting rationally to the limited information that they have~\cite{Kobes:2009jx,Schadschneider:2008cz,Reicher:2008ep,Torres:2010tj,Paulsen:1984ti,Sime:1983uy}. Ignoring this perceived \emph{irrationality} reduces the reliability of these models.



% Recently, experts in various fields~\cite{Arthur:2010uy} are emphasizing the importance of accepting that most systems in the world are complex systems and the need to model these complexities to have a realistic chance at understanding and predicting what happens in the real world. 



% One of the primary reasons for early models like lattice gas models~\cite{Takima:2002wr} and flow models~\cite{Henderson:1974ve} abstracting away details was the  and getting an approximate result in order to get such details as are necessary for that particular application. Advances in hardware have removed many of these constraints. As a result, models have increasingly become more detailed and capable~\cite{Pan:2006vp}. 

There have been tremendous advancements in our knowledge of human behavior over the last few decades. However, the majority of the state of the art computational models make assumptions about human behavior without grounding them in the theories and findings from social sciences. Being a very inter-disciplinary field this kind of collaboration is absolutely essential. 

\section{Problem Statement}
\label{Intro:ProblemStatement}

Several studies over the past few decades have changed our understanding of human behavior during fire evacuations. Some~\cite{Kobes:2009jx,Schadschneider:2008cz,Reicher:2008ep,Torres:2010tj,Paulsen:1984ti,Sime:1983uy} have shown how humans always behave rationally with the limited information that they have and that humans hardly ever panic and behave irrationally. Others have highlighted the effect of stress and time constraints on human behavior~\cite{Ozel:2001tn}. Torres's thesis~\cite{Torres:2010tj} compared and analyzed the effectiveness of various theories in explaining a real life fire scenario. Aguirre~\cite{Aguirre:2004tn} gave an excellent criticism of existing computational models of egress and the shortcomings and strengths of different models.

However, existing computational models make several assumptions about human behavior that contradict these studies. Pre-evacuation behavior and the search for information are two general characteristics of the fire evacuation process that are very rarely considered in egress simulation models. The few computational models of egress that do consider pre-evacuation~\cite{Pires:2005gs,Klupfel:2003wa} behavior are rather simple. In other cases, like the models of human motion and exploration used in existing models, there are several different approaches that have been used but a consistent methodology for their evaluation and validation is lacking. 


This thesis aims at enhancing the capabilities of existing computation models of egress by proposing novel methods to evaluate and improve core aspects of existing models, and to model certain important aspects that haven't been given their due importance in existing models. This is done with the help of a new generic agent architecture, which we call the \emph{Information Based EVACuation~(IBEVAC) Model}.

% What are the key components of my approach and results? Also include any specific limitations. The goals you are going to achieve;% The tasks to complete in order to attain the goals, or the direction of the thesis research development;

\section{Key Contributions and Scope of this Thesis}
\label{Intro:Contributions}

Some of the key contributions of this thesis are:
\begin{itemize}
	\item A multidisciplinary survey and analysis of current literature on fire evacuation and crowd behavior.

	\item A detailed quantitative comparison between popular models for movement. The methods used in this comparison provides a very useful tool for comparing existing and future models of crowd movement. The results of comparing some of the most popular current models of movement.

	\item A novel information based perception system that can model the complexities and limitations of the human perception system: This model was presented at the CyberWorlds 2011 conference.

	\item A novel model of pre-evacuation behavior and the process of information seeking: This phase of evacuation has been proven to exist by several studies but has been generally ignored in most computational models of egress. A novel computational model for modeling pre evacuation behavior was implemented and presented at the Pedestrian and Evacuation Dynamics 2012 conference in Zurich.

	\item A human computation based investigation of the nature of human spatial memory and human exploration of multi-storey buildings. This analysis produced some interesting results that has been submitted to  .....  . 

	

\end{itemize}


\section{Organization of the Report}
\label{Intro:Organisation}

 This report is organized as follows. 

 % Chapter~\ref{chapter:LiteratureReview} provides a comprehensive review of relevant theories and pre-existing models and also provides a critical analysis of some relevant ones. Having presented the current state of the art, the following chapter provides an overview of the overall Information Based EVACuation (IBEVAC) model architecture. Chapter~\ref{chapter:IBP} then presents the new Information Based Perception Model and some experiments that demonstrate its capabilities and working. Chapter~\ref{chapter:TheRemainingModules} gives an introduction to the proposed structure and working of the remaining modules of the architecture. Finally, Chapter~\ref{chapter:ConclusionAndFutureWork} winds up the report by presenting, in brief, the work that remains to be done and a plan of action for the period of the author's candidature.


