%!TEX root = vaisagh_thesis.tex
\chapter{Conclusion and Future Work}
\label{chapter:finale}

Crowd egress from buildings is a complex process. Understanding and analyzing the behavior of crowds during egress is crucial in order to prevent disasters like the Love Parade and the Iroquois Theater from being repeated. Modeling and simulation is a primary tool for doing this. Modern agent based models, in particular, can have complex behavior models that enable the simulation and analysis of egress in much higher detail than was ever possible. However, computational models are still unable to simulate the entire process of egress from the sound of the fire alarm to the last person evacuating safely. This thesis took a holistic approach to understanding and modelling the egress process. Literature was examined from all related fields, including psychology, civil engineering, sociology, etc. Along with this understanding, incident reports and analyses were used to clearly identify the core components of the egress process and discover the steps in that process that are critical to improving safety. Chapter~\ref{chapter:IBEVAC} desribes the details of the egress process gained from this research. With this fundamental understanding it was then possible to evaluate the existing computational models commonly used when modelling human egress. This comparison lead to the identification of key research questions in which either new modelling approach or new understanding was necessary. The contributions of this thesis was to develop these areas and thereby improve the state of the art in the computational modelling of egress. In what follows the identified research gaps are summarised.

 The commonly held conception of emergency egress is that during an emergency, people panic, stampede and head for the closest exit and in the process injure and kill others. However, the consensus among crowd behavior experts who've studied past incidents is that people do not \emph{panic} or behave irrationally in a crowd (see chapter~\ref{chapter:LiteratureReview}). People generally tend to behave rationally within the bounds of their knowledge. This has two important implications for egress modelling: Firstly, modelling rational, thinking human beings would be necessary for realistic egress simulation; secondly, the complete process of how people gain information from the environment and react to it will have to be considered to model behavior during egress. Simply considering agents that rush straight for the exit at the sound of the fire alarm (which is generally considered the start of an evacuation simulation) would be insufficient to accurately simulate egress.

 A study of incident reports, psychology and sociology studies and experiments in psychology (see Chapter~\ref{chapter:LiteratureReview}) revealed a great deal about human behavior during egress. Occupants rarely have a complete understanding of the situation when an emergency starts. Most occupants do not have complete knowledge of the spatial layout of the building or the best routes for escape. In fact, many studies report the significant amount of time that is spent by occupants, engaging in pre-evacuation behavior of first disbelieving the first cues that are observed and subsequently investigating for more information. This search for information is further complicated by the effect of stress and time constraints which limit the information processing capacity of the brain. The human brain compensates for this by chucking together similar information and only processing what is determined to be the most significant and important of these cues. Once the need to evacuate is determined, occupants try to search for familiar and reassuring surroundings by trying to find their family or friends, i.e. their primary group. Once found, they try to exit by the shortest route \emph{that they know of}. This is further complicated by the fact that most occupants do not know the complete layout of the building. This leads to inefficient routes and exploration behavior. Eventually, through following others and, generally, with the help of trained personnel occupants escape from the building.

 In general, the study of human behavior literature seemed to indicate that most of the process of human behavior during egress is well understood. Several computation models of varying complexity have been developed over the years to simulate this egress behavior. In order to study these models, they were categorized according to the approach used for modeling behavior and the level of detail in which the crowd is treated. Agent based models, with their approach to considering individual agents and their interactions, were found to come closest to simulating the complex process of egress behavior. Several highly capable models exist that are able to simulate the effect of emotion, leadership behavior and the impact of fire and smoke on mobility and visibility. However, existing models did not even attempt at simulating pre-evacuation behavior or the effects of human cognitive limits on other aspects of human behavior like spatial knowledge utilization and movement. In the few cases, that this was attempted, it was observed that the models were too simple to help in actually analyzing the effects of these information processing limits on egress.
 % In some cases like the modeling of spatial knowledge utilization and movement modeling, new methods had to be developed


% What was the aim of the thesis?
% Agent based simulations of crowds are an invaluable aid in preparing for events, planning for emergencies and even in games and movies. Despite there existing a large body of work addressing the different aspects of crowd simulation there are still several shortcomings in even the most advanced of today's models.
% The aim of this thesis was to advancing the state of the art in the computational modelling of egress by first identifying the key components of a behavior model of egress and subsequently proposing models to identify, simulate and analyze the effect of human cognitive limits on these components.

% What did literature show?

% Much research has been done in different fields on human behavior during egress. There have been several surveys of specific aspects of egress behavior like crowd behavior and pre-evacuation investigation like~\cite{Torres:2010tj, Kuligowski:2009un}. This thesis began with a broader multidisciplinary overview of our current understanding of emergency egress behavior. This shed light on parts of egress behavior that have been scientifically established but are generally ignored in computational models. Firstly, several studies showing that humans do not behave irrationally or \emph{panic} during evacuation were discussed. Generally, people behave rationally within the bounds of their knowledge. What generally appears as panic during post hoc analysis is simply a consequence of the fact that sufficient information was not available to the occupants at the time~\cite{Helbing:2012}. This gave a hint to the primary role that information acquisition and processing plays in egress process. This was further confirmed by the extensive literature that discusses the pre-evacuation phase where occupants search for \emph{cues} that give them more information about the situation and establish a need to evacuate. Research has shown that this phase of pre-evacuation can last several minutes and is one of the common causes of deaths during egress. Once evacuation starts people tend to make use of their existing knowledge as much as possible by taking familiar routes to the exit rather than the shortest ones. This is further complicated by the fact that, once some route is blocked, many  occupants may not have sufficient knowledge to plan a route to the exit. Studies have shown that even in such scenarios people tend to try to evacuate rather than hide or fight the fire. This leads to people exploring indoor environments in search of the information they require: a route to escape and safety. The limited information processing capacity of the human short term memory was demonstrated by~\cite{Miller:1956tr} more than half a century ago. Moreover, there has been varying amounts of research in cognitive sciences, psychology and sociology of the working of short term memory and various factors like the effect of stress, time constraints, old age, informative cues, etc. on the  cognitive limits of our working memory. In order to prevent human tragedy caused by badly managed egress from closed spaces, it is necessary that planners take these factors into account when preparing for events or constructing buildings.

% Modelling and simulation is the primary tool for preparing for these disasters and has been for some time now. The second part of Chapter~\ref{chapter:LiteratureReview} discussed the different approaches that have been used over the years for simulating egress behavior. Until fairly recently, models did not attempt to simulate individual behavior and rather focused on producing and analyzing the macro level dynamics of the crowd. A more detailed analysis of microscopic models that model individual behavior gave an idea of the capabilities of existing models. The effectiveness of simple rules in producing complex emergent behavior, the importance of well structured agent architectures and different methods for modelling perception, communication, navigation and human behavior were all analyzed. Most importantly, it was observed that, in spite of the understanding of the important role played by information processing in the egress process, it is almost never taken into consideration even in detailed computational models of egress.


% As a first step towards

% A review of the literature showed that people do not behave irrationally or \emph{panic} during an emergency. Rather people behave rationally within the bounds of their knowledge. A key part of the emergency egress process is the \emph{pre-evacuation} phase during which occupants perceive the first cue that suggests egress might be required and try to determine the need to evacuate. Following this they try to evacuate the building by using their existing knowledge of the layout of the building. The lack of complete knowledge leads to people taking routes that are most familiar to them rather than what might actually be the shortest. Other aspects of emergency egress behavior like grouping and the effect of stress and time constraints were also discussed.

%mhl : First, we examined the literature studying real world egress, in which scientists have analysed many incidents to ascertain patterns of behavior and dynamics. Then we studied the existing models to see how they map to what the real domain experts are saying. From this, we developed a research plan to help understand a few things about egress: What is teh role of information in way fining in motion planning and in pre-evacuation. How do we assess existing models of motion. NO standard here.


% Following this, a review of existing models of emergency egress simulation was done to determine the contributions the thesis could make. First a broad overview of existing models was given based on the type of approach used starting from the earliest network flow models to the much more complicated agent based approaches of today. Some of the most advanced and most significant crowd simulation models were then examined in more detail to get a better idea of the existing state of the art and opportunities for research.


% THe four building blocks that were identified


As a first step to modelling and analyzing the effect of human cognitive limits on egress behavior, the four key components of the egress process were identified: Perception, Event Identification, Spatial Knowledge Utilization and Navigation.
There are several issues that arise in modelling and analyzing the effect of cognitive limits on each of these components. In some cases, like \emph{event identification} and \emph{perception}, there is lot of existing literature in psychology and cognitive science on the effects of these cognitive limits; however, they have never been considered or modeled in existing computational models. In these cases, Chapters~\ref{chapter:IBP} and~\ref{chapter:PreEvacuationBehavior} proposed novel models for these and demonstrated the usefulness of considering these in egress modelling. In other cases, like \emph{spatial knowledge modelling}, a study of the literature revealed that we do not yet have a sufficient understanding of exploration behavior in the presence of lack of spatial information and the role of memory in wayfinding. Thus, in Chapter~\ref{chapter:SpatialKnowledgeChapter}, a scalable game based methodology was developed to learn more about the role of memory and the strategies used by people in indoor wayfinding. Finally, in the case of \emph{navigation} modelling, there exist a plethora of models taking different approaches to navigation; however, a standard methodology for comparing and analyzing these different models has not been developed. This makes it difficult to determine the effectiveness or usefulness of a new model even if it is developed. This research problem was addressed in Chapter~\ref{chapter:MotionPlannerComparison}. In the following sections of this final chapter, the work done in each of these chapters is briefly reviewed with their limitations and future possibilities.

\section{Modelling Human Cognitive Limits in Perception} % (fold)
\label{sec:an_information_processing_based_approach_to_perception}

The first component of simulating human behavior in egress is the model of how the environment is perceived. Perception is a fundamental part of any agent based model of crowds. Existing crowd simulation models simply consider perception as agents perceiving all objects within a certain circular (or at times an elliptic) sensing range. However, as mentioned earlier, limitations of human short term memory mean that a person is able to process no more than nine percepts at any given time. In fact, more recently, it's been shown that this limit is probably closer to five than nine. One of the ways in which the human mind counters this limitation is by chunking together similar information thus making better use of its limited capacity. We believed that modeling this is necessary to get more realistic simulations of human behavior.

In Chapter~\ref{chapter:IBP}, a model for perception that takes into consideration this limited information processing capacity and chunking of information was introduced. The novel idea was in considering sensing as a process of gathering \emph{information} from the environment rather that simply visual percepts for collision avoidance. Its working and usefulness was demonstrated by enhancing the popular Reciprocal Velocity Obstacle Based motion planning system with this Information-processing Based Perception (IBP) System. The chunking of information was modeled through a global dynamic clustering that is performed once per time-step. Through this efficient group based perception system, the way in which people avoid groups of people coming towards them was modeled. Multiple layers of clustering were used to ensure that simulated pedestrians did not move unnaturally when avoiding big groups. Comparison against real world experiments and analysis of simulation results in several standard scenarios showed the modeling perception as a process of gathering information helped produce more realistic results. Simulations were also used to show how a navigation system using IBP could work in a system where the limited information processing capacity of humans was considered. In principle, IBP could be used to model all forms of perception of an agent while still taking into consideration the cognitive limits of short term memory.



% \subsection{Limitations and future work} % (fold)
% \label{sec:limitations_ibp}

One of the primary aims of the development of IBP was to combine it with a cue perception system while still taking into consideration the limited information processing capacity of human beings. A similar filtering system could be used for determining the information value of cue percepts and other percepts and only a limited number of the most important percepts would be processed by the agents. Theoretically, this is likely to produce more realistic behavior and also the bounded rationality that is discussed in literature on behavior during emergencies. However, there are several factors that are very difficult to consider in doing this, the most pertinent of which would be the difficulty in assigning an information value to different percepts.


% A further question that might be asked of the work is whether it is reasonable to assume that groups of people can be equated with Miller's \emph{chunks}. I believe that the value that this abstraction provides in improving the capability of existing models makes it an invaluable approach regardless of the equivalence.

Ideally, the proposed model would have been tested against more video evidence of crowds. However, this data is not easy to obtain and validate against. Another limitation of the work, due to time constraints, is that most of the simulations performed were rather simplistic compared to the multi floor building simulations in Chapter~\ref{chapter:PreEvacuationBehavior}.

% Also, in theory, the IBP model can be used for other motion planning systems like Social Force. However, this was not tested or demonstrated. It would be interesting to test the difficulties in adapting this system to other motion planning systems.

% section limitations (end)


Perhaps, one of the most interesting products of this research is the basis it provides for modeling cognitive limits in computational models of humans. The idea of grouping together similar information during perception to model scientifically observed \emph{chunking} behavior provides us with better insight into the implications of the extensive work done in cognitive science. In the context of egress simulation, it provides planners with a method for analyzing and determining how information can be passed to evacuees during emergencies where our ability to perceive information from the environment becomes even more constrained due to stress and time constraints.  In the following chapter, a computational model for how occupants process perceived information during egress was developed.

% section an_information_processing_based_approach_to_perception (end)

\section{A Novel Model for Event Identification} % (fold)
\label{sec:a_novel_model_for_pre_evacuation_behavior}

% It is a commonly experienced fact that most people don't start evacuating when they first hear a fire alarm. The first instinct is to assume it is false. Several studies have shown that this is practically always the case. In fact, after some delay, people start investigating in order to confirm the need to evacuate and only when the need is confirmed do they actually start evacuating. However, despite all the evidence, most simulations assume (mostly for simplicity) that people start evacuating immediately. At best, a small time delay is added before the agents start evacuating.

In Chapter~\ref{chapter:PreEvacuationBehavior}, a method for modelling the processing of cues and identifying events was described. It was demonstrated how this model could be used to model pre-evacuation behavior and the effect of different kinds of fire alarms and other cues. Experiments demonstrated the capabilities of the model and, more importantly, identified the necessity of modelling of pre-evacuation behavior. For example, they helped identify the critical number of occupants that need to be trained if they were randomly distributed over a two storey environment. The impact of milling behavior on evacuation dynamics was also shown. The model was also used to show that strategically placed, trained personnel may be just as effective as an expensive fire alarm in improving survival rate.


% \subsection{Limitations and future work} % (fold)
% \label{sec:limitations_pre_evac}

This bucket based model of event identification opens up several possibilities for future work. Some of these could be done through modifications to the experiments conducted and were suggested in the chapter itself. For example, the model could be used for identifying strategies for distributing trained personnel. A preliminary exploration into the usefulness of strategic distribution was in fact presented as well. Different strategies of exploration for trained personnel could be explored to ensure maximum survival or the amount of time they should spend finding others before evacuating themselves could be determined if the building structure is known. The experiment on milling behavior could be used during building construction to ensure optimality of milling/ gathering positions. In case of budget constraints, the optimal distribution of expensive and unambiguous fire alarm systems and cheaper more ambiguous alarms could be determined while ensuring maximum effectiveness.

The major limitation of this study was the lack of comparison against real world data. Considering the nature of the scenarios under consideration, there is little that can be done to counter this. However, with the advent and popularity of wearable electronic devices and the array of sensors on modern smartphones there is a possibility that this data might be available in the future. At present, the best possible solution would be to use a game based methodology like the one proposed in Chapter~\ref{chapter:SpatialKnowledgeChapter}.

As discussed in Section~\ref{sec:sec:an_information_processing_based_approach_to_perception}, another avenue of future work would be to combine the cue perception and event identification system with the Information Based Perception system and to analyze the effectiveness of this combined model. This may also help identify further shortcomings in both these models. More importantly, it would open doors to using modeling and simulation to analyze how information can be passed to occupants to improve egress efficiency; for example, more complex fire alarm systems and interactive sign boards could be tested and improved cheaply before a huge investment is made. This kind of analysis, however, is difficult to do with our current understanding of the effect of partial spatial knowledge on indoor wayfinding. This was further explored in the following chapter.

% In the future, the effect of more complex cues like perception of running agents when there is no alarm or the lack of perception of running agents when there is an alarm could be explored with experiments similar to the ones used here.

% section future_work (end)

% section a_novel_model_for_pre_evacuation_behavior (end)

\section{A Game Based Analysis of Indoor Wayfinding Behavior} % (fold)
\label{sec:a_game_based_analysis_of_indoor_wayfinding_behavior}

As mentioned earlier, one of the difficulties in modelling the effect of the limits of human information processing capacity on spatial knowledge utilization is our limited understanding of the phenomenon.
Existing models of simulation take the simple approach of assuming that all evacuees have complete knowledge of the environment. When partial knowledge is modeled, a simple approach of having perfect memory and remembering all rooms perceived once is modeled. Also, exploration behavior has rarely been explored in any detail in existing crowd simulation models. A review of existing literature in cognitive psychology revealed that a lot of work had been done on human spatial memory and exploration of outdoor environments. However, exploration and way-finding in multi-floor indoor environments is still not very well understood. It was also discovered that while interesting experiments have been developed to study indoor wayfinding, these experiments are difficult to scale and tend to analyze only a few participants. This makes it difficult to do more detailed analysis.

To explore and understand the role of memory in indoor wayfinding, a game was developed in Minecraft which required players to explore a multi-floor building and complete certain tasks within certain time constraints. Using Minecraft had several advantages including the popularity of the game, the ease of changing the environment and, most importantly, the possibility of tracking all the actions of players. This data was then analysed using some agent based models to determine strategies of exploration and spatial memory limitations. It was found that, during exploration, people prefer a floor first strategy and try to avoid making decisions to change direction unless they are forced to. A novel Markovian analysis was developed to analyze the size of memory used during exploration by the players and revealed that $6-8$ step memory is sufficient to produce human like exploration in agents.



% \subsection{Limitations and future work} % (fold)
% \label{sec:limitations_wayfinding}

Due to the game being hosted on a local server in NTU, the dataset size of much lesser than expected. This made it difficult to do what if analysis against alternate scenarios which would be required for getting stronger conclusions and verifying hypothesis. Currently, data is being collected from hosting the game online and this can hopefully be used for making stronger conclusions. A larger dataset would also allow for doing the Markov analysis for larger values of $m$.

A limitation of the scaled experiment being hosted online is the dependence on the integrity of the players. It is near impossible to check for all kind of cheating. Also it is more difficult to do traditional methods like asking players to speak their thoughts out loud to determine their thought process. A more complicated incentive system which pays players some amount based on their achievements may have to be developed to ensure participation.

In spite of these limitations, even on a local server, the game based methodology was able to gather a larger participant base than existing desktop virtual reality experiments. This proved invaluable in using the Markov Agent Based analysis which provided novel insights into the role of short term memory in indoor wayfinding.


% section future_work (end)

% section a_game_based_analysis_of_indoor_wayfinding_behavior (end)
\section{A Comparison of Existing Motion Planning Systems} % (fold)
\label{sec:a_comparison_of_existing_motion_planning_systems}

The fourth and final component of the egress process is the navigation system. This part of the model simulates how the agent moves from a particular source location to it's destination while avoiding collisions with other agents. The navigation system can generally be divided into two parts: a higher level path planning system and a lower level motion planning system. The path planning system determines a high level path towards a selected goal, usually in terms of some way-points. The motion planning system which is sometimes also called a collision avoidance system helps the agent select a collision free velocity while moving towards the next way-point. There has been a lot of research on this and there are several different motion planning systems that exist today. A problem that lies in determining or developing the most realistic motion planning system among these is the lack of a method for the systematic comparison of these different motion planning systems. In Chapter~\ref{chapter:MotionPlannerComparison}, a methodology for comparing different models of crowds was introduced and the DISTATIS method was used to identify zoned evacuation time as a key metric that can be used in differentiating the different models. Video footage from the Sichuan Earthquake in China was then used for determining which model produced the most \emph{realistic} behavior.


% Contribution 4 : Motion Planner Comparision

% \subsection{Limitations and future work} % (fold)
% \label{sec:limitations_motion_planners}

The simulations in this research were done only in a simple square room with a single exit. Similar analysis in more complex layouts may help reveal more about the models being compared. Also, only three of the most popular models i.e. RVO2, Social Force and Lattice Gas models were compared in this study. More models will probably need to be compared to increase confidence in the notion of zoned evacuation time being the key metric for differentiation between models.

Due to the limited amount of video footage available and the difficulty in extracting meaningful statistics from these videos, only the flow rate could be determined for the real world data and compared against the models. Even though zoned evacuation time was identified as the key metric, it was impossible to measure this from the limited video data available.

Other metrics that weren't considered in the paper could also be used for comparing the models and a similar DISTATIS method can be used for discovering a new key metric. We do not hypothesize that the methodology proposed is the final arbiter on deciding which motion planning system to use. When actually using a motion planning system other considerations will have to be made as well like the purpose of the simulation, the computational resources required, the ease of development and its compatibility with the other systems in the simulation. However, the methodology developed does provide a novel and systematic method for comparison of different models and validating new models of navigation that are developed.

% section future_work (end)
% section a_comparison_of_existing_motion_planning_systems (end)

\section{Conclusion} % (fold)
\label{sec:conclusion_of_conclusion}

Over the course of this thesis methods that enable the modeling and analysis of the effect of human cognitive limits on the egress process has been developed. This chapter has reviewed the major contributions of this thesis with brief discussions of the several possible extensions and new areas of research that have opened up as a result of this thesis. The work done in this thesis not only advanced the state of the art in the computational modelling of egress but has also helped improve our understanding of the different components of human behavior during emergency egress. This could prove invaluable in saving lives by helping planners understand, manage and control dangerous situations that develop with crowding.
% section conclusion (end)